\documentclass[a4paper,10pt]{article}
\usepackage[utf8]{inputenc}

\usepackage{amsmath}
\usepackage{amssymb}
\usepackage[square,numbers]{natbib}
\bibliographystyle{unsrtnat}

\title{Deep Learning for High Schoolers}
\author{Sharad Vikram}

\begin{document}

\maketitle

\section{Introduction}
Artificial intelligence dominated
the headlines when Google Deepmind's
AlphaGo defeated Go professional 
Lee Sedol. The event was a testament
to the progress of artificial intelligence
research over the past decade.
One of the main components that 
enabled AlphaGo to win,
however, has been around for
more than 30 years, namely
the neural network.
Over the past few years,
neural network research
has had a resurgence,
resulting in a new
area of research for artificial
intelligence, called \emph{deep learning}.

In this writeup, I explain
from the ground up the core mathematics
and algorithms
used in deep learning.
This writeup will act as lecture notes
for a presentation to Torrey Pines High School
math students, hence the name.
The prerequisites are
a basic understanding of multivariable calculus
and linear algebra.

\section{Background}
Artificial intelligence is a broad
term that encompasses several areas of
research, but in general,
the goal of artificial intelligence
is to write computer programs
that exhibit intelligent behavior.
There are several approaches
to writing such a computer program.
<describe approaches>

In this writeup, we focus on
\emph{machine learning},
where computers
learn patterns from
data to perform a task.

\subsection*{What is machine learning?}
Machine learning is the subfield of
artificial intelligence
dealing with learning from data.
Machine learning itself can be split up
into
supervised and unsupervised learning.

In supervised learning,
the computer is provided
data in the form of \emph{labeled examples},
or input-output pairs,
and is tasked with creating
some function that maps inputs
to outputs, while also generalizing
to never before seen inputs.
In unsupervised learning,
the problem is much less specified.
The computer is provided with data
in the form of just inputs,
and is tasked with discovering
underlying patterns in the inputs.

Unsupervised learning is considered
much harder than supervised learning.
Consider the classic supervised learning problem 
of determining if an email is spam or not.
Our program is provided with
a set of emails, annotated with whether they are spam or not.
It's not unreasonable that a computer program performs
some sort of word association, to figure out
which words are more associated with spam emails
and which are not.
However, now consider the same task,
but we do not know which emails are spam beforehand.
An unsupervised learning algorithm's goal
would be to discover the properties of
spammy emails, without knowledge of which ones
are spam in the first place.

In this writeup, we concentrate on a particular
flavor of supervised learning algorithms,
neural networks. Machine learning with
``deep neural networks'', or deep learning 
as it is called today, has skyrocketed
in popularity.

\section{Supervised Learning}
The supervised learning problem can be expressed very concisely.
Given a set of input-output pairs $\{(x_1, y_1)\, (x_2, y_2), \ldots, (x_N, y_N)\}$,
return a function $h(x)$ that when given an
input $x$, predicts its corresponding output $y$.
In general, all $x$'s are vectors of identical dimension,
but the form of $y$ can vary from task to task.

Supervised learning problems can be broken down into
two closely related tasks, \emph{classification} and \emph{regression},
both of which relate to the type of prediction at hand.
In classification, we want a function $h(x)$ that, from an input,
predicts a category, for example
deciding whether an email is spam or not. Each input
can only belong to a finite set of categories. Other examples
are facial recognition, where an picture of a face is mapped to a person,
and song recognition, mapping snippets of noisy audio to the song they 
originated from.
In regression, we want $h(x)$ to output a real number. Examples of
regression tasks include extrapolating stock market prices 
and predicting atmospheric gas levels from noisy sensor readings.
Although these two tasks are inherently different,
the algorithms for each are very closely related to each other.

Now, consider the following scenario. You're provided
with a of input-output pairs $\{(x_1, y_1)\, (x_2, y_2), \ldots, (x_N, y_N)\}$,
which we'll call a \emph{dataset}.
How do we even begin coming up with an $h(x)$? A simple approach
would be to ``memorize'' the dataset. When we pass an $x$
into $h(x)$, it simply looks up the known examples
in the dataset for a matching $x_i$, and returns the corresponding
$y_i$. Although this approach is simple, it fails at \emph{generalizing}.
Given an $x$ not in the dataset, $h(x)$ will not know what to do.
We can't possibly expect this $h(x)$ to work in a real-life
scenario with unexpected events and noise.

An approach to better generalize 
is to pick a functional form, or \emph{model},
of $h(x)$. 
There are so many possible options for $h(x)$,
so we typically start by constraining the
possible types of functions that $h(x)$ could be.
For example, we could assert that $h(x)$ must be
a \emph{linear function}, that is,
if $x$ is a vector,
\begin{align}
  h(x) &= w^Tx + b
\end{align}
Now, the problem of finding 
best $h(x)$ is much easier.
Instead of deciding on the best $h(x)$
to fit the data from the space of all possible
functions, we just need to
find the best $w$ and $b$,
a task called \emph{linear regression}.
We'll discuss exactly how find such a $w$ and $b$ later.

Of course, asserting that $h(x)$
is linear has downsides.
For example, a linear function is ill-suited
for any classification tasks,
since linear functions map real-valued inputs
to a real-valued output.
Moreover, if the relationship
between the inputs and outputs
is far more complex than linear,
our linear $h(x)$ would not be a great predictor.

%Thus, we want to open the door to 
%more complex functional forms.
%A logical step up is
%the $\emph{quadratic function}$.
%For input vector $x$, our function
%is constrained to the form
%\begin{align}
%h(x) = x^TAx + b^Tx + c
%\end{align}
%The ``unknowns''
%are matrix $A$, vector $b$ and scalar $c$.
%Amazingly, any function
%that could have been produced
%by linear regression,
%could be produced by quadratic regression (plug in 
%a matrix of zeros for $A$).

We can imagine using more complex functions.
Maybe a quadratic or a cubic
or a 10th degree polynomial
will fit the data better?
Why wouldn't we then always use
a more complex model?
Why wouldn't we always use quadratic regression
over linear regression? There are a couple
issues that come up when we make our functions more complicated.
In linear regression, where we had to
only decide values $w$ and $b$, a vector
and a scalar, we now have to decide
values for a matrix, a vector,
and a scalar. We now see a tradeoff
with choosing a complicated functional form.
As our functions grow more complex,
we need to learn more unknowns from the data.

The linear regression model isn't
naturally built for classification.
We can make it work though,
by thresholding.
First, imagine we're doing
binary classification,
where we're deciding between two
classes, $0$ and $1$.
We could take the output of our linear $h(x)$,
and predict class $1$ if it's positive
and class $0$ otherwise.
What if we can make this linear classifier
even better? Let's make it so instead of
$h(x)$ predicting an arbitrary real value,
it outputs
the \emph{probability} of an input $x$
belonging to class $1$.
If the output is greater than $0.5$,
then $P(\text{class}\; 1) > P(\text{class}\; 0)$
and we'll output class $1$,
and otherwise, we'll output class $0$.
In order to do this, we need to force
$h(x)$ to output a number between 0 and 1,
requiring the use of the \emph{logistic function}.

The logistic function is
\begin{align}
  \theta(z) = \frac{1}{1 + e^{-z}}
\end{align}
It has the effect of \emph{squashing} its
input if it gets too high or low.
As $z \rightarrow \infty$, $\theta(z) \rightarrow 1$,
and as $z \rightarrow -\infty, \theta(z) \rightarrow 0$.
Furthermore, $\theta(0) = 0.5$.
Let's take the output of a linear function of $x$
and pass it into the logistic function.
We now get the model
\begin{align}
  h(x) = \frac{1}{1 + e^{-(w^Tx + b)}}
\end{align}
This function has the nice benefit of being a 
linear classifier (if $w^Tx + b > 0$, then
$P(\text{class} \; 1) > 0.5$),
while also outputting a probability.
It is a natural fit for binary classification,
and the model is called \emph{logistic regression}.
We'll soon see that logistic regression is a
core building block of the neural network.

\section{Optimization}

\section{Neural Networks}
Let's dive right into neural networks.
Neural networks are a framework
for building functions
for both classification and regression tasks,
\emph{inspired} by the biology of the brain.
Let's quickly discuss how the brain works.
The brain consists of cells called
\emph{neurons}, which can be roughly divided
into two parts, the dendrite and the axon.
A dendrite receives signals from other neighboring neurons,
combines them, and 
fires a signal through its axon if the combined
input signal is strong enough.
A neuron will saturate if
the input signal is strong enough.
This means that the neuron's output
caps as the input gets stronger.

This idea of receiving inputs,
combining them, and sending an output
that saturates
is the basis for what we call
the \emph{artificial neuron}.
The artificial neuron is an extremely
rough model for a real neuron,
so by no means is a neural
network ``simulating'' the brain.
The artificial neuron is a building block
for a neural network just as the neuron
is a building block for the brain.

Every neuron outputs a scalar number,
and receives input from several other neurons
in the form of a vector $x$.
The output of the neuron $f(x)$ combines
the inputs in the following way:

\begin{align*}
  f(x) = \theta(w \cdot x + b)
\end{align*}

In this equation, $\theta$
is some function we choose beforehand
and $w$ and $b$ are unknowns as before.
$f(x)$ looks oddly similar to our
logistic regression function,
if we just choose $\theta$ 
to be the logistic function.
In fact, a single artificial neuron
is a generalization of logistic regression.
Each element of the vector $x_i$
represents the output of some
other artificial neuron.
We take a linear combination of the inputs
as we do in linear regression, but then we apply
$\theta$, a \emph{non-linear} function.
This non-linear function often saturates,
just like the logistic function,
where the output converges as
the input gets extremely positive or negative.
Again, just like linear regression and
logistic regression, the ``unknowns''
in our function are $w$ and $b$.

%Why is it crucial that $\theta$ be non-linear?
%Imagine we were to string a bunch of neurons
%together without applying $\theta$.
%We would repeatedly be taking
%linear combinations of an input $x$.
%The problem is that multiple
%linear combinations of $x$
%can be represented by a single one,
%so any number of linear neurons
%could be equivalently represented by 
%just one.

The next big difference between
a neural network and the brain
is how we connect neurons together.
In the real brain,
there are cycles in neuron connectivity
(neuron A can connect to neuron B which
can connect back to A). In
neural networks, we only allow
\emph{feed-forward} connections.
Neurons are only allowed to connect
forwards, towards the desired output
and there are no cycles allowed.
It may seem like we're removing
a critical component that makes the brain powerful,
but cycles can make it almost impossible
to do anything practical. We'll see why later.

How do feed-forward connections work?
It's easiest to see them layer-wise.
Imagine the input to our network, $x$ (the data
we wish to predict an output for),
as a series of stacked nodes.
Each node feeds into a layer of stacked artificial neurons.
Each of these artificial neurons performs its calculation (a linear combination
and a non-linear function), and has its own output.
This output itself can be fed into another layer of
stacked neurons. Eventually, the final layer of our network
will represent the network's prediction $y$.
We can pick and choose how many layers
there are in the network, and how many neurons
are in each layer. This is where
the power of neural networks comes in. Remember
that there could be scenarios where
a simple linear or logistic function isn't
powerful enough. To make a neural network
more complex, we simply make the layers
bigger or add more layers.

Let's look at a neural network
in more mathematical detail.
So each neuron
performs a dot product, adds a scalar bias
term, and then applies a non-linearity.
Imagine that each neuron in a layer
uses the same non-linear function.
We can now vectorize
the entire layer's computation!
Each neuron receives the previous layer's output $x$
as input.
Let's say we have $d$ neurons in the current layer.
The output of the $i$-th neuron
in the layer can be written as:
\begin{align*}
  o_i = \theta(w_i \cdot x + b_i)
\end{align*}
Since each neuron outputs a scalar,
we can take each of the $d$ outputs
and make a $d$-dimensional vector
$o = [o_1, \ldots, o_d]$.
It turns out that $o$ can be written as
\begin{align*}
  o = \theta(Wx + b)
\end{align*}
where $W$ is a matrix
whose rows are each $w_i$
and $b$ is a vector
whose entries are each $b_i$.
$W$ is $d_o \times d_i$,
where $d_o$ is the size of our layer 
and $d_i$ is dimension
of the input vector.

Each layer will perform a similar
calculation. Let's assume
for simplicity, every layer in the network
uses the same non-linear function $\theta$.
For a 2-layer network,
our final $h(x)$ can be written compactly as
\begin{align*}
  h(x) = \theta(W_2\theta(W_1x + b_1) + b_2)
\end{align*}

Adding a layer to the network nests the output
of one network in yet another
linear combination and non-linearity.

What non-linear functions do we typically use?
Well, the logistic is perhaps the simplest one.
Another popular one is the hyperbolic tangent function,
or \emph{tanh}. It looks like
\begin{align*}
  \theta(z) = \frac{e^z - e^{-z}}{e^z + e^{-z}}
\end{align*}

The tanh behaves exactly like the logistic function,
but saturates at $-1$ and 1, instead of 0 and 1.

\bibliography{dlhs}

\end{document}
